\documentclass{article}
\usepackage[utf8]{inputenc}
%packages used
\usepackage[left=3cm, right=3cm, top=3cm]{geometry}
\usepackage[utf8]{inputenc}
\usepackage{amsmath, amssymb}
\usepackage{graphicx}
\usepackage{hyperref}
\usepackage{enumerate}
\usepackage[hang,flushmargin]{footmisc}
\usepackage{amsfonts}
\usepackage{amsthm}
\usepackage{xcolor}
\usepackage{graphicx}
\usepackage{centernot}
\usepackage{multicol}
\usepackage{color}
\usepackage{verbatim}
\usepackage{caption}
\usepackage{subcaption}
\usepackage{epsfig}
\graphicspath{ {./images/} }

\newtheorem{theorem}{Theorem}
\newtheorem{lemma}{Lemma}
\newtheorem{corollary}{Corollary}
\newtheorem{definition}{Definition}

%newly defined commands
\renewcommand{\today}{}
\newcommand{\abs}[1]{\left| #1\right|}
\newcommand{\Lap}[1]{\mathcal{L}\left\{#1\right\}}
\newcommand{\solution}[1]{
    \color{red}\begin{quote}Solution:\quad 
    \color{black} #1
    \end{quote}\color{black}
}
\newcommand{\ba}{\backslash}
\newcommand{\Ber}{\hbox{Ber}}
\newcommand{\p}[1]{\mathbb{P}\left(#1\right)}
\newcommand{\e}[1]{\mathbb{E}(#1)}
\newcommand{\Po}[1]{\hbox{Po}(#1)}
\newcommand{\var}[1]{\hbox{Var}(#1)}
\newcommand{\Z}{\mathbb{Z}}
\newcommand{\R}{\mathbb{R}}
\newcommand{\Q}{\mathbb{Q}}
\newcommand{\N}{\mathbb{N}}
\newcommand{\floor}[1]{\left\lfloor #1\right\rfloor}
\newcommand{\C}{\mathbb{C}}
\DeclareMathOperator{\Diam}{diam}
\newcommand{\diam}[1]{\Diam\left(#1\right)}
\DeclareMathOperator{\nul}{Null}
\newcommand{\Null}[1]{\nul\left(#1\right)}
\DeclareMathOperator{\Dimension}{dim}
\newcommand{\Dim}[1]{\Dimension\left(#1\right)}
\newcommand{\threevec}[3]{\left[\begin{array}{r} #1 \\ #2 \\ #3\end{array}\right]}
\newcommand{\fourvec}[4]{\left[\begin{array}{r} #1 \\ #2 \\ #3\\#4\end{array}\right]}
\DeclareMathOperator{\Rank}{rank}
\newcommand{\rank}[1]{\Rank\left(#1\right)}
\DeclareMathOperator{\Nullity}{nullity}
\newcommand{\nullity}[1]{\Nullity\left(#1\right)}
\DeclareMathOperator{\Sp}{Span}
\renewcommand{\sp}[1]{\Sp\left\{#1\right\}}
\DeclareMathOperator{\col}{Col}
\newcommand{\Col}[1]{\col\left(#1\right)}
    

 

\title{Markov Chains}
\author{Akshath Jain \& Deepayan Patra}
\date{December 2018}

\usepackage{natbib}
\usepackage{graphicx}

\begin{document}

\maketitle

\section{Overview}

For the Kids: Snakes and Ladders

High-School Example: Populations and Predator-Prey Relationships

Markov-Chain Monte Carlo Methods

Chemical Reaction Mechanisms

\section{A Brief History of Markov Chains and Page-Rank}

\section{Referenced Definitions and Theorems}
    \begin{definition}{Markov Chain}
    
    \end{definition}
    
    \begin{definition}{Stochastic Matrix}
    An $n \times n$ matrix is stochastic given its entries represent 
    \end{definition}
    
    \begin{definition}{Altered Transition Matrix}
    
    \end{definition}
    
    \begin{definition}{Steady-State/Equilibrium Vector}
    
    \end{definition}
    
    \begin{definition}{Altered Transition Matrix}

    \end{definition}
    
    \begin{definition}{Dangling Node}

    \end{definition}
    
    \begin{definition}{Damping Constant}

    \end{definition}
    
    \begin{definition}{Irreducible Graph}

    \end{definition}
    
    \begin{definition}{Reducible Graph}

    \end{definition}
    
    \begin{theorem}{Perron-Frobenius Theorem}
    
    \end{theorem}
\section{Markov Chain Algorithm}

\subsection{Adjacency Matrix Construction}

\section{Working Example}

\section{Conclusion}

\bibliographystyle{plain}
\bibliography{references}
http://mathworld.wolfram.com/StochasticMatrix.html
\end{document}
